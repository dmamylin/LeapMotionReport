\documentclass{report}

\usepackage[utf8]{inputenc}
\usepackage[russian]{babel}
\usepackage{gensymb}

\begin{document}
    \section*{Слайд 1}
        Leap Motion, карта глубин и OpenCV\\
        Тут пока хз что...

    \section*{Слайд 4}
        \begin{itemize}
			\item 2 камеры с линзами fish-eye
			\item 3 инфракрасных светодиода
			\item Интерфейс: USB 3.0 micro-B
			\item Дальность распознавания объектов $\approx$ 60 см
			\item Угол обзора $\approx$ 135$\degree$
			\item Есть стандартный SDK, который периодически обновляется
			\item Монохромные изображения с разрешением 620x240 на выходе
		\end{itemize}
		
	\section*{Слайд 6}
	    То есть, в отличие от настоящих 3D-сканеров, Leap Motion
	    не создает карту глубин, а для распознавания рук применяются
	    специальные алгоритмы. Единственная информация, которая нас
	    интересует - изображения. По сути, это просто две камеры.
	    Хорошо, пусть так, тогда для получения карты глубин можно
	    использовать сторонние библиотеки, например OpenCV.
	    Но и такой подход не дает нужного результата, поскольку:
	    
	\section*{Слайд 7}
    	\begin{itemize}
			\item Камеры очень чувствительны к освещению
			\item Отражения света от светодиодов распознаются как объекты
		\end{itemize}
	    Из-за этого карта глубин получается зашумленной.
	    
	    К сожалению, эти недостатки были замечены поздно. Зато был отработан
	    подход, который можно применить к любой другой системе объемного зрения.
\end{document}
